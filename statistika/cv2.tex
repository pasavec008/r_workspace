% Options for packages loaded elsewhere
\PassOptionsToPackage{unicode}{hyperref}
\PassOptionsToPackage{hyphens}{url}
%
\documentclass[
]{article}
\usepackage{amsmath,amssymb}
\usepackage{lmodern}
\usepackage{iftex}
\ifPDFTeX
  \usepackage[T1]{fontenc}
  \usepackage[utf8]{inputenc}
  \usepackage{textcomp} % provide euro and other symbols
\else % if luatex or xetex
  \usepackage{unicode-math}
  \defaultfontfeatures{Scale=MatchLowercase}
  \defaultfontfeatures[\rmfamily]{Ligatures=TeX,Scale=1}
\fi
% Use upquote if available, for straight quotes in verbatim environments
\IfFileExists{upquote.sty}{\usepackage{upquote}}{}
\IfFileExists{microtype.sty}{% use microtype if available
  \usepackage[]{microtype}
  \UseMicrotypeSet[protrusion]{basicmath} % disable protrusion for tt fonts
}{}
\makeatletter
\@ifundefined{KOMAClassName}{% if non-KOMA class
  \IfFileExists{parskip.sty}{%
    \usepackage{parskip}
  }{% else
    \setlength{\parindent}{0pt}
    \setlength{\parskip}{6pt plus 2pt minus 1pt}}
}{% if KOMA class
  \KOMAoptions{parskip=half}}
\makeatother
\usepackage{xcolor}
\usepackage[margin=1in]{geometry}
\usepackage{color}
\usepackage{fancyvrb}
\newcommand{\VerbBar}{|}
\newcommand{\VERB}{\Verb[commandchars=\\\{\}]}
\DefineVerbatimEnvironment{Highlighting}{Verbatim}{commandchars=\\\{\}}
% Add ',fontsize=\small' for more characters per line
\usepackage{framed}
\definecolor{shadecolor}{RGB}{248,248,248}
\newenvironment{Shaded}{\begin{snugshade}}{\end{snugshade}}
\newcommand{\AlertTok}[1]{\textcolor[rgb]{0.94,0.16,0.16}{#1}}
\newcommand{\AnnotationTok}[1]{\textcolor[rgb]{0.56,0.35,0.01}{\textbf{\textit{#1}}}}
\newcommand{\AttributeTok}[1]{\textcolor[rgb]{0.77,0.63,0.00}{#1}}
\newcommand{\BaseNTok}[1]{\textcolor[rgb]{0.00,0.00,0.81}{#1}}
\newcommand{\BuiltInTok}[1]{#1}
\newcommand{\CharTok}[1]{\textcolor[rgb]{0.31,0.60,0.02}{#1}}
\newcommand{\CommentTok}[1]{\textcolor[rgb]{0.56,0.35,0.01}{\textit{#1}}}
\newcommand{\CommentVarTok}[1]{\textcolor[rgb]{0.56,0.35,0.01}{\textbf{\textit{#1}}}}
\newcommand{\ConstantTok}[1]{\textcolor[rgb]{0.00,0.00,0.00}{#1}}
\newcommand{\ControlFlowTok}[1]{\textcolor[rgb]{0.13,0.29,0.53}{\textbf{#1}}}
\newcommand{\DataTypeTok}[1]{\textcolor[rgb]{0.13,0.29,0.53}{#1}}
\newcommand{\DecValTok}[1]{\textcolor[rgb]{0.00,0.00,0.81}{#1}}
\newcommand{\DocumentationTok}[1]{\textcolor[rgb]{0.56,0.35,0.01}{\textbf{\textit{#1}}}}
\newcommand{\ErrorTok}[1]{\textcolor[rgb]{0.64,0.00,0.00}{\textbf{#1}}}
\newcommand{\ExtensionTok}[1]{#1}
\newcommand{\FloatTok}[1]{\textcolor[rgb]{0.00,0.00,0.81}{#1}}
\newcommand{\FunctionTok}[1]{\textcolor[rgb]{0.00,0.00,0.00}{#1}}
\newcommand{\ImportTok}[1]{#1}
\newcommand{\InformationTok}[1]{\textcolor[rgb]{0.56,0.35,0.01}{\textbf{\textit{#1}}}}
\newcommand{\KeywordTok}[1]{\textcolor[rgb]{0.13,0.29,0.53}{\textbf{#1}}}
\newcommand{\NormalTok}[1]{#1}
\newcommand{\OperatorTok}[1]{\textcolor[rgb]{0.81,0.36,0.00}{\textbf{#1}}}
\newcommand{\OtherTok}[1]{\textcolor[rgb]{0.56,0.35,0.01}{#1}}
\newcommand{\PreprocessorTok}[1]{\textcolor[rgb]{0.56,0.35,0.01}{\textit{#1}}}
\newcommand{\RegionMarkerTok}[1]{#1}
\newcommand{\SpecialCharTok}[1]{\textcolor[rgb]{0.00,0.00,0.00}{#1}}
\newcommand{\SpecialStringTok}[1]{\textcolor[rgb]{0.31,0.60,0.02}{#1}}
\newcommand{\StringTok}[1]{\textcolor[rgb]{0.31,0.60,0.02}{#1}}
\newcommand{\VariableTok}[1]{\textcolor[rgb]{0.00,0.00,0.00}{#1}}
\newcommand{\VerbatimStringTok}[1]{\textcolor[rgb]{0.31,0.60,0.02}{#1}}
\newcommand{\WarningTok}[1]{\textcolor[rgb]{0.56,0.35,0.01}{\textbf{\textit{#1}}}}
\usepackage{graphicx}
\makeatletter
\def\maxwidth{\ifdim\Gin@nat@width>\linewidth\linewidth\else\Gin@nat@width\fi}
\def\maxheight{\ifdim\Gin@nat@height>\textheight\textheight\else\Gin@nat@height\fi}
\makeatother
% Scale images if necessary, so that they will not overflow the page
% margins by default, and it is still possible to overwrite the defaults
% using explicit options in \includegraphics[width, height, ...]{}
\setkeys{Gin}{width=\maxwidth,height=\maxheight,keepaspectratio}
% Set default figure placement to htbp
\makeatletter
\def\fps@figure{htbp}
\makeatother
\setlength{\emergencystretch}{3em} % prevent overfull lines
\providecommand{\tightlist}{%
  \setlength{\itemsep}{0pt}\setlength{\parskip}{0pt}}
\setcounter{secnumdepth}{-\maxdimen} % remove section numbering
\ifLuaTeX
  \usepackage{selnolig}  % disable illegal ligatures
\fi
\IfFileExists{bookmark.sty}{\usepackage{bookmark}}{\usepackage{hyperref}}
\IfFileExists{xurl.sty}{\usepackage{xurl}}{} % add URL line breaks if available
\urlstyle{same} % disable monospaced font for URLs
\hypersetup{
  pdftitle={cv2.R},
  pdfauthor={Marek},
  hidelinks,
  pdfcreator={LaTeX via pandoc}}

\title{cv2.R}
\author{Marek}
\date{2023-02-23}

\begin{document}
\maketitle

\begin{Shaded}
\begin{Highlighting}[]
\CommentTok{\#Diskretne a spojite rozdelenia pravdepodobnosti}
\CommentTok{\#Kazde rozdelenie ma svoj zakladny prikaz a k nemu dane}
\CommentTok{\#pismeno, podla toho, co ratame}
\CommentTok{\# d ratame P(x=k)}
\CommentTok{\# p ratame P(X\textbackslash{}leq k)}
\CommentTok{\# q ratame kvantily}
\CommentTok{\# r generujeme nahodne cislo z daneho rozdelenia}
\CommentTok{\# Binomicke rozdelenie}
\CommentTok{\# V statistickej kontrole akosti, n/krat nezavisle na sebe opakujeme}
\CommentTok{\# pokus, sledujeme vyskyt danej udalosti}
\CommentTok{\# Pravdepodobnost danes udalosti v jedinom pokuse je p,}
\CommentTok{\# parametre su n,p, binom}

\CommentTok{\# Pr.1}
\CommentTok{\# Ucinost antibiotika je  80\%. Podavame ho 10 pacientom na oddeleni.}
\CommentTok{\# Zratajte tieto pravdepodobnosti:}
\CommentTok{\# Vsetci pacienti sa vyliecia P(x=10)}
\CommentTok{\# Prave 7 sa vylieci P(x=7)}
\CommentTok{\# Najviac 8 sa vylieci P(x\textless{}=8).}
\CommentTok{\# Aspon 5 sa vylieci P(x\textgreater{}=5) = P(x\textgreater{}4)  ostra nerovnost je lepsia}

\NormalTok{n }\OtherTok{\textless{}{-}} \DecValTok{10}
\NormalTok{p }\OtherTok{\textless{}{-}} \FloatTok{0.8}
\FunctionTok{dbinom}\NormalTok{(}\DecValTok{10}\NormalTok{, n, p) }\CommentTok{\#prva uloha}
\end{Highlighting}
\end{Shaded}

\begin{verbatim}
## [1] 0.1073742
\end{verbatim}

\begin{Shaded}
\begin{Highlighting}[]
\FunctionTok{dbinom}\NormalTok{(}\DecValTok{7}\NormalTok{, n, p) }\CommentTok{\#druha uloha}
\end{Highlighting}
\end{Shaded}

\begin{verbatim}
## [1] 0.2013266
\end{verbatim}

\begin{Shaded}
\begin{Highlighting}[]
\FunctionTok{pbinom}\NormalTok{(}\DecValTok{8}\NormalTok{, n, p) }\CommentTok{\#tretia uloha}
\end{Highlighting}
\end{Shaded}

\begin{verbatim}
## [1] 0.6241904
\end{verbatim}

\begin{Shaded}
\begin{Highlighting}[]
\FunctionTok{pbinom}\NormalTok{(}\DecValTok{4}\NormalTok{, n, p, }\AttributeTok{lower.tail =}\NormalTok{ F) }\CommentTok{\#stvrta uloha {-} prva moznost}
\end{Highlighting}
\end{Shaded}

\begin{verbatim}
## [1] 0.9936306
\end{verbatim}

\begin{Shaded}
\begin{Highlighting}[]
\DecValTok{1} \SpecialCharTok{{-}} \FunctionTok{pbinom}\NormalTok{(}\DecValTok{4}\NormalTok{, n, p) }\CommentTok{\#stvrta uloha {-} druha moznost}
\end{Highlighting}
\end{Shaded}

\begin{verbatim}
## [1] 0.9936306
\end{verbatim}

\begin{Shaded}
\begin{Highlighting}[]
\CommentTok{\# DO tej hodnoty je neostra, OD tej hodnoty je ostra}
\CommentTok{\# pbinom ked kumulujem}

\CommentTok{\#Zostrojte tabulku a graf rozdelenia pravdepodobnosti}
\NormalTok{xB }\OtherTok{\textless{}{-}} \DecValTok{0}\SpecialCharTok{:}\DecValTok{10}
\NormalTok{hustotaB }\OtherTok{\textless{}{-}} \FunctionTok{dbinom}\NormalTok{(xB, n, p)}
\NormalTok{hustotaB}
\end{Highlighting}
\end{Shaded}

\begin{verbatim}
##  [1] 0.0000001024 0.0000040960 0.0000737280 0.0007864320 0.0055050240
##  [6] 0.0264241152 0.0880803840 0.2013265920 0.3019898880 0.2684354560
## [11] 0.1073741824
\end{verbatim}

\begin{Shaded}
\begin{Highlighting}[]
\NormalTok{(tabulkaB }\OtherTok{\textless{}{-}} \FunctionTok{data.frame}\NormalTok{(}\AttributeTok{hodnota =}\NormalTok{ xB, }\AttributeTok{pravdepodobnost =}\NormalTok{ hustotaB))}
\end{Highlighting}
\end{Shaded}

\begin{verbatim}
##    hodnota pravdepodobnost
## 1        0    0.0000001024
## 2        1    0.0000040960
## 3        2    0.0000737280
## 4        3    0.0007864320
## 5        4    0.0055050240
## 6        5    0.0264241152
## 7        6    0.0880803840
## 8        7    0.2013265920
## 9        8    0.3019898880
## 10       9    0.2684354560
## 11      10    0.1073741824
\end{verbatim}

\begin{Shaded}
\begin{Highlighting}[]
\NormalTok{tabulkaB}
\end{Highlighting}
\end{Shaded}

\begin{verbatim}
##    hodnota pravdepodobnost
## 1        0    0.0000001024
## 2        1    0.0000040960
## 3        2    0.0000737280
## 4        3    0.0007864320
## 5        4    0.0055050240
## 6        5    0.0264241152
## 7        6    0.0880803840
## 8        7    0.2013265920
## 9        8    0.3019898880
## 10       9    0.2684354560
## 11      10    0.1073741824
\end{verbatim}

\begin{Shaded}
\begin{Highlighting}[]
\FunctionTok{View}\NormalTok{(tabulkaB)}
\FunctionTok{barplot}\NormalTok{(tabulkaB}\SpecialCharTok{$}\NormalTok{pravdepodobnost, }\AttributeTok{main =} \StringTok{\textquotesingle{}Binomicke rozdelenie\textquotesingle{}}\NormalTok{, }\AttributeTok{names.arg =}\NormalTok{ xB,}
        \AttributeTok{xlab =} \StringTok{\textquotesingle{}hodnota\textquotesingle{}}\NormalTok{, }\AttributeTok{ylab =} \StringTok{\textquotesingle{}pravdepodobnost\textquotesingle{}}\NormalTok{, }\AttributeTok{col =} \StringTok{\textquotesingle{}green\textquotesingle{}}\NormalTok{)}
\end{Highlighting}
\end{Shaded}

\includegraphics{cv2_files/figure-latex/unnamed-chunk-1-1.pdf}

\begin{Shaded}
\begin{Highlighting}[]
\CommentTok{\# hodnoty distribucnej funkcie}
\NormalTok{(distB }\OtherTok{\textless{}{-}} \FunctionTok{pbinom}\NormalTok{(xB, n, p))}
\end{Highlighting}
\end{Shaded}

\begin{verbatim}
##  [1] 0.0000001024 0.0000041984 0.0000779264 0.0008643584 0.0063693824
##  [6] 0.0327934976 0.1208738816 0.3222004736 0.6241903616 0.8926258176
## [11] 1.0000000000
\end{verbatim}

\begin{Shaded}
\begin{Highlighting}[]
\DocumentationTok{\#\#\#\#\#\#\#\#\#\#\#\#\#\#\#\#\#\#\#\#\#\#\#\#\#\#\#\#\#\#\#\#\#\#\#\#\#\#\#\#\#\#\#\#\#\#\#}
\CommentTok{\# Hypergeometricke rozdelenie}
\CommentTok{\# Pouzitie v statistickej kontrole akosti. Mnozina obsahuje}
\CommentTok{\# m{-}prvkov so sledovanou vlastnostou a}
\CommentTok{\# n{-}prvkov bez tejto vlastnosti,}
\CommentTok{\# nahodne vyberieme k{-}prvkov.}
\CommentTok{\# Nahodna premenna je pocet prvkov so sledovanou vlastnostou v nasom vybere.}
\CommentTok{\# Prikaz je hyper, pismenka platia, parametre v poradi m, n, k}

\CommentTok{\# Priklad}
\CommentTok{\# Student sa nauci na skusku 12 z 20 otazok. Test obsahuje 5 otazok.}
\CommentTok{\# Vypocitajte nasledujuce pravdepodobnost:}
\CommentTok{\# Student dostane Acko, zodpovie vsetky otazky P(x=5)}
\CommentTok{\# Student neurobi skusku, zodpovie menej ako 3 P(x\textless{}=2)  pre dolnu hranicu neostra}
\CommentTok{\# Student urobi skusku (doplnok k predchadzajucej), zodpovie 3 a viac P(x\textgreater{}2)}
\CommentTok{\# Parametre su m = 12, n = 8, k = 5}
\NormalTok{m }\OtherTok{\textless{}{-}} \DecValTok{12}
\NormalTok{n }\OtherTok{\textless{}{-}} \DecValTok{8}
\NormalTok{k }\OtherTok{\textless{}{-}} \DecValTok{5}
\CommentTok{\# prva uloha}
\FunctionTok{dhyper}\NormalTok{(}\DecValTok{5}\NormalTok{, m, n, k)}
\end{Highlighting}
\end{Shaded}

\begin{verbatim}
## [1] 0.05108359
\end{verbatim}

\begin{Shaded}
\begin{Highlighting}[]
\CommentTok{\# druha uloha}
\FunctionTok{phyper}\NormalTok{(}\DecValTok{2}\NormalTok{, m, n, k)}
\end{Highlighting}
\end{Shaded}

\begin{verbatim}
## [1] 0.2961816
\end{verbatim}

\begin{Shaded}
\begin{Highlighting}[]
\CommentTok{\# tretia uloha}
\DecValTok{1} \SpecialCharTok{{-}} \FunctionTok{phyper}\NormalTok{(}\DecValTok{2}\NormalTok{, m, n, k) }\CommentTok{\# ako doplnok}
\end{Highlighting}
\end{Shaded}

\begin{verbatim}
## [1] 0.7038184
\end{verbatim}

\begin{Shaded}
\begin{Highlighting}[]
\FunctionTok{phyper}\NormalTok{(}\DecValTok{2}\NormalTok{, m, n, k, }\AttributeTok{lower.tail =}\NormalTok{ F)}
\end{Highlighting}
\end{Shaded}

\begin{verbatim}
## [1] 0.7038184
\end{verbatim}

\begin{Shaded}
\begin{Highlighting}[]
\CommentTok{\# tabulka a graf rozdelenia pravdepodobnosti}
\NormalTok{xH }\OtherTok{\textless{}{-}} \DecValTok{0}\SpecialCharTok{:}\DecValTok{5} \CommentTok{\# viem zodpovedat ani jednu az vsetkych 5 otazok}
\NormalTok{hustotaH }\OtherTok{\textless{}{-}} \FunctionTok{dhyper}\NormalTok{(xH, m, n, k)}
\NormalTok{tabulkaH }\OtherTok{\textless{}{-}} \FunctionTok{data.frame}\NormalTok{(}\AttributeTok{hodnota =}\NormalTok{ xH, }\AttributeTok{pravdepodobnost =}\NormalTok{ hustotaH)}
\NormalTok{tabulkaH}
\end{Highlighting}
\end{Shaded}

\begin{verbatim}
##   hodnota pravdepodobnost
## 1       0     0.003611971
## 2       1     0.054179567
## 3       2     0.238390093
## 4       3     0.397316821
## 5       4     0.255417957
## 6       5     0.051083591
\end{verbatim}

\begin{Shaded}
\begin{Highlighting}[]
\FunctionTok{View}\NormalTok{(tabulkaH)}
\FunctionTok{barplot}\NormalTok{(}
\NormalTok{  tabulkaH}\SpecialCharTok{$}\NormalTok{pravdepodobnost,}
  \AttributeTok{main =} \StringTok{\textquotesingle{}Hypergeometricke rozdelenie\textquotesingle{}}\NormalTok{,}
  \AttributeTok{names.arg =}\NormalTok{ xH,}
  \AttributeTok{xlab =} \StringTok{\textquotesingle{}hodnota\textquotesingle{}}\NormalTok{,}
  \AttributeTok{ylab =} \StringTok{\textquotesingle{}pravdepodobnost\textquotesingle{}}\NormalTok{,}
  \AttributeTok{col =} \StringTok{\textquotesingle{}blue\textquotesingle{}}\NormalTok{,}
  \AttributeTok{ylim =} \FunctionTok{c}\NormalTok{(}\DecValTok{0}\NormalTok{,}\FloatTok{0.4}\NormalTok{)}
\NormalTok{)}
\end{Highlighting}
\end{Shaded}

\includegraphics{cv2_files/figure-latex/unnamed-chunk-1-2.pdf}

\begin{Shaded}
\begin{Highlighting}[]
\CommentTok{\# nakreslite empiricku distribucnu funkciu pomocou nasimulovanych dat 10 000}
\NormalTok{data }\OtherTok{\textless{}{-}} \FunctionTok{rhyper}\NormalTok{(}\DecValTok{10000}\NormalTok{, m, n, k)}
\FunctionTok{plot}\NormalTok{(}\FunctionTok{ecdf}\NormalTok{(data), }\AttributeTok{main =} \StringTok{\textquotesingle{}Empiricka distribucna funkcia\textquotesingle{}}\NormalTok{)}
\end{Highlighting}
\end{Shaded}

\includegraphics{cv2_files/figure-latex/unnamed-chunk-1-3.pdf}

\begin{Shaded}
\begin{Highlighting}[]
\DocumentationTok{\#\#\#\#\#\#\#\#\#\#\#\#\#\#\#\#\#\#\#\#\#\#\#\#\#\#\#\#\#\#\#\#\#\#\#\#\#\#\#\#\#\#}
\CommentTok{\# Poissonovo rozdelenie}
\CommentTok{\# pouziva sa v teorii hromadnej obsluhy, pravdepodobnosti zriedkavych javov}
\CommentTok{\# v casovom intervale, na nejakom objeme. Ma jediny parameter lambda}
\CommentTok{\# Pri zmene casoveho intervalu treba parameter tiez prepocitat.}
\CommentTok{\# Lambda je ocakavana hodnota v zadani.}
\CommentTok{\# Prikaz je pois}
\CommentTok{\# Na samoobsluznu linku pride 20 ludi za hodinu. Vypocitajte tieto pravdepodobnosti:}
\CommentTok{\# V priebehu 15 min. pride 1 clovek P(x = 1), prepocet lambda = 20/60 je jedna minuta}
\CommentTok{\# 15 = 5}
\CommentTok{\# V priebehu 5 min. nikto nepride P(x = 0), lmbda = 20/60, 5 = 5 / 3}
\CommentTok{\# V priebehu 10 min. pride aspon 10 ludi, P(x \textgreater{}= 10) = P(x \textgreater{} 9),}
\CommentTok{\# lambda = 20/60, 10 = 510/3}

\CommentTok{\#prva uloha}
\FunctionTok{dpois}\NormalTok{(}\DecValTok{1}\NormalTok{, }\DecValTok{5}\NormalTok{)}
\end{Highlighting}
\end{Shaded}

\begin{verbatim}
## [1] 0.03368973
\end{verbatim}

\begin{Shaded}
\begin{Highlighting}[]
\CommentTok{\#druha uloha}
\FunctionTok{dpois}\NormalTok{(}\DecValTok{0}\NormalTok{, }\DecValTok{5}\SpecialCharTok{/}\DecValTok{3}\NormalTok{)}
\end{Highlighting}
\end{Shaded}

\begin{verbatim}
## [1] 0.1888756
\end{verbatim}

\begin{Shaded}
\begin{Highlighting}[]
\CommentTok{\#tretia uloha}
\FunctionTok{ppois}\NormalTok{(}\DecValTok{9}\NormalTok{, }\DecValTok{10}\SpecialCharTok{/}\DecValTok{3}\NormalTok{, }\AttributeTok{lower.tail =}\NormalTok{ F)}
\end{Highlighting}
\end{Shaded}

\begin{verbatim}
## [1] 0.002356375
\end{verbatim}

\begin{Shaded}
\begin{Highlighting}[]
\CommentTok{\# Uvazujme casovy okamih 1 hodinu, urcte maximalny pocet ludi, ktori navstivia}
\CommentTok{\# linku s pravdepodobnostou 90\% (na linku pride max. tolko ludi) .. zarucujeme }
\CommentTok{\# sa, ze na 90\% tolko pride max}
\FunctionTok{qpois}\NormalTok{(}\FloatTok{0.9}\NormalTok{, }\DecValTok{20}\NormalTok{)}
\end{Highlighting}
\end{Shaded}

\begin{verbatim}
## [1] 26
\end{verbatim}

\begin{Shaded}
\begin{Highlighting}[]
\CommentTok{\# zostrojte tabulku a graf rozdelenia pravdepodobnosti}
\CommentTok{\# pre interval 1 hodina a prvych 40 hodnot}
\NormalTok{xP }\OtherTok{\textless{}{-}} \DecValTok{0}\SpecialCharTok{:}\DecValTok{40} \CommentTok{\# viem zodpovedat ani jednu az vsetkych 5 otazok}
\NormalTok{hustotaP }\OtherTok{\textless{}{-}} \FunctionTok{dpois}\NormalTok{(xP, }\DecValTok{20}\NormalTok{)}
\NormalTok{tabulkaP }\OtherTok{\textless{}{-}} \FunctionTok{data.frame}\NormalTok{(}\AttributeTok{hodnota =}\NormalTok{ xP, }\AttributeTok{pravdepodobnost =}\NormalTok{ hustotaP)}
\NormalTok{tabulkaP}
\end{Highlighting}
\end{Shaded}

\begin{verbatim}
##    hodnota pravdepodobnost
## 1        0    2.061154e-09
## 2        1    4.122307e-08
## 3        2    4.122307e-07
## 4        3    2.748205e-06
## 5        4    1.374102e-05
## 6        5    5.496410e-05
## 7        6    1.832137e-04
## 8        7    5.234676e-04
## 9        8    1.308669e-03
## 10       9    2.908153e-03
## 11      10    5.816307e-03
## 12      11    1.057510e-02
## 13      12    1.762517e-02
## 14      13    2.711565e-02
## 15      14    3.873664e-02
## 16      15    5.164885e-02
## 17      16    6.456107e-02
## 18      17    7.595420e-02
## 19      18    8.439355e-02
## 20      19    8.883532e-02
## 21      20    8.883532e-02
## 22      21    8.460506e-02
## 23      22    7.691369e-02
## 24      23    6.688147e-02
## 25      24    5.573456e-02
## 26      25    4.458765e-02
## 27      26    3.429819e-02
## 28      27    2.540607e-02
## 29      28    1.814719e-02
## 30      29    1.251530e-02
## 31      30    8.343536e-03
## 32      31    5.382927e-03
## 33      32    3.364329e-03
## 34      33    2.038987e-03
## 35      34    1.199404e-03
## 36      35    6.853739e-04
## 37      36    3.807633e-04
## 38      37    2.058180e-04
## 39      38    1.083253e-04
## 40      39    5.555141e-05
## 41      40    2.777571e-05
\end{verbatim}

\begin{Shaded}
\begin{Highlighting}[]
\FunctionTok{View}\NormalTok{(tabulkaP)}
\FunctionTok{barplot}\NormalTok{(}
\NormalTok{  tabulkaP}\SpecialCharTok{$}\NormalTok{pravdepodobnost,}
  \AttributeTok{main =} \StringTok{\textquotesingle{}Poissonovo rozdelenie\textquotesingle{}}\NormalTok{,}
  \AttributeTok{names.arg =}\NormalTok{ xP,}
  \AttributeTok{xlab =} \StringTok{\textquotesingle{}hodnota\textquotesingle{}}\NormalTok{,}
  \AttributeTok{ylab =} \StringTok{\textquotesingle{}pravdepodobnost\textquotesingle{}}\NormalTok{,}
  \AttributeTok{col =} \StringTok{\textquotesingle{}red\textquotesingle{}}\NormalTok{,}
  \AttributeTok{ylim =} \FunctionTok{c}\NormalTok{(}\DecValTok{0}\NormalTok{,}\FloatTok{0.15}\NormalTok{)}
\NormalTok{)}
\end{Highlighting}
\end{Shaded}

\includegraphics{cv2_files/figure-latex/unnamed-chunk-1-4.pdf}

\begin{Shaded}
\begin{Highlighting}[]
\DocumentationTok{\#\#\#\#\#\#\#\#\#\#\#\#\#\#\#\#\#\#\#\#\#\#\#\#\#\#\#\#\#\#\#\#}
\CommentTok{\# spojite rozdelenia}
\CommentTok{\# normalne rozdelenie pravdepodobnosti ma dva parametre, strednu hodnotu}
\CommentTok{\# mu (to je asi nejakej pismenko) a smerodajnu odchylku sigma}
\CommentTok{\# prikaz je norm}
\CommentTok{\# Zivotnost bateriek do mobilnych telefonov sa riadi normalnym rozdelenim}
\CommentTok{\# so strednou hodnotou 8 a smerodajnou odchylkou 2}
\CommentTok{\# Ulohy}
\CommentTok{\# Kolko \% bateriek treba vymenit do 7.5 roka P(x \textless{}= 7.5)}
\CommentTok{\# Kolko \% bateriek vydrzi v rozpati 7{-}9 rokov P(7 \textless{}= x \textless{}= 9)}
\CommentTok{\# Kolko vydrzi viac ako 10 P(x \textgreater{}= 10)}
\CommentTok{\# Za aku dobu zivotnosi sa mozno zarucit na 90\% (tolko a viac)}

\CommentTok{\# prva uloha}
\FunctionTok{pnorm}\NormalTok{(}\FloatTok{7.5}\NormalTok{, }\AttributeTok{mean =} \DecValTok{8}\NormalTok{, }\AttributeTok{sd =} \DecValTok{2}\NormalTok{)}
\end{Highlighting}
\end{Shaded}

\begin{verbatim}
## [1] 0.4012937
\end{verbatim}

\begin{Shaded}
\begin{Highlighting}[]
\CommentTok{\# druha uloha}
\CommentTok{\# na dva kroky, najskor spocitam po a potom odcitam}
\FunctionTok{pnorm}\NormalTok{(}\DecValTok{9}\NormalTok{, }\AttributeTok{mean =} \DecValTok{8}\NormalTok{, }\AttributeTok{sd =} \DecValTok{2}\NormalTok{) }\SpecialCharTok{{-}} \FunctionTok{pnorm}\NormalTok{(}\DecValTok{7}\NormalTok{, }\AttributeTok{mean =} \DecValTok{8}\NormalTok{, }\AttributeTok{sd =} \DecValTok{2}\NormalTok{)}
\end{Highlighting}
\end{Shaded}

\begin{verbatim}
## [1] 0.3829249
\end{verbatim}

\begin{Shaded}
\begin{Highlighting}[]
\CommentTok{\# tretia uloha}
\FunctionTok{pnorm}\NormalTok{(}\DecValTok{10}\NormalTok{, }\AttributeTok{mean =} \DecValTok{8}\NormalTok{, }\AttributeTok{sd =} \DecValTok{2}\NormalTok{, }\AttributeTok{lower.tail =}\NormalTok{ F)}
\end{Highlighting}
\end{Shaded}

\begin{verbatim}
## [1] 0.1586553
\end{verbatim}

\begin{Shaded}
\begin{Highlighting}[]
\CommentTok{\# stvrta uloha}
\FunctionTok{qnorm}\NormalTok{(}\FloatTok{0.9}\NormalTok{, }\AttributeTok{mean =} \DecValTok{8}\NormalTok{, }\AttributeTok{sd =} \DecValTok{2}\NormalTok{, }\AttributeTok{lower.tail =}\NormalTok{ F) }\CommentTok{\#zarucujem sa na tolko a hornu hranicu}
\end{Highlighting}
\end{Shaded}

\begin{verbatim}
## [1] 5.436897
\end{verbatim}

\begin{Shaded}
\begin{Highlighting}[]
\CommentTok{\# nakreslime histogram nasimulovanych dat, N(0, 1), prelozime hustotu cez histogram}
\NormalTok{xx }\OtherTok{\textless{}{-}} \FunctionTok{rnorm}\NormalTok{(}\DecValTok{500}\NormalTok{, }\AttributeTok{mean =} \DecValTok{0}\NormalTok{, }\AttributeTok{sd =} \DecValTok{1}\NormalTok{) }\CommentTok{\# pre histogram}
\NormalTok{xxx }\OtherTok{\textless{}{-}} \FunctionTok{seq}\NormalTok{(}\SpecialCharTok{{-}}\DecValTok{3}\NormalTok{, }\DecValTok{3}\NormalTok{, }\FloatTok{0.01}\NormalTok{) }\CommentTok{\# pre kreslenie hustoty}
\FunctionTok{hist}\NormalTok{(xx, }\AttributeTok{freq =}\NormalTok{ F) }\CommentTok{\# freq F na zmenu mierky na pravdepodobnostnu}
\FunctionTok{lines}\NormalTok{(xxx, }\FunctionTok{dnorm}\NormalTok{(xxx, }\DecValTok{0}\NormalTok{, }\DecValTok{1}\NormalTok{), }\AttributeTok{col =} \StringTok{\textquotesingle{}blue\textquotesingle{}}\NormalTok{)}
\end{Highlighting}
\end{Shaded}

\includegraphics{cv2_files/figure-latex/unnamed-chunk-1-5.pdf}

\begin{Shaded}
\begin{Highlighting}[]
\DocumentationTok{\#\#\#\#\#\#\#\#\#\#\#\#\#\#\#\#\#\#\#\#\#\#\#\#\#\#\#\#\#\#\#\#\#\#\#\#\#\#\#\#\#\#}
\CommentTok{\# Exponencialne rozdelenie}
\CommentTok{\# Zivotnost zariadenia, doba do prvej poruchy, doby medzi poruchami,}
\CommentTok{\# jediny parameter lambda, je to prevratena hodnota strednej hodnoty,}
\CommentTok{\# pozor ako to bude v zadani. E(x) = 1 / lambda}
\CommentTok{\# Priklad}
\CommentTok{\# Dlzka zivotnosti pouzivaneho PC v pocitacovej ucebni je 2 roky}
\CommentTok{\# Vypocitajte PC ma zivotnost aspon 1 rok P(X \textgreater{}= 1)}
\CommentTok{\# PC ma zivotnost najviac \% rokov P(X \textless{}= 5)}
\CommentTok{\# Za aku zivotnost by ste sa zarucili s pravdepodobnostou 5\%, p = 0.05}
\CommentTok{\# parameter je 1/2}
\CommentTok{\# Uloha 1}
\FunctionTok{pexp}\NormalTok{(}\DecValTok{1}\NormalTok{, }\AttributeTok{rate =} \DecValTok{1}\SpecialCharTok{/}\DecValTok{2}\NormalTok{, }\AttributeTok{lower.tail =}\NormalTok{ F)}
\end{Highlighting}
\end{Shaded}

\begin{verbatim}
## [1] 0.6065307
\end{verbatim}

\begin{Shaded}
\begin{Highlighting}[]
\CommentTok{\# Uloha 2}
\FunctionTok{pexp}\NormalTok{(}\DecValTok{5}\NormalTok{, }\AttributeTok{rate =} \DecValTok{1}\SpecialCharTok{/}\DecValTok{2}\NormalTok{)}
\end{Highlighting}
\end{Shaded}

\begin{verbatim}
## [1] 0.917915
\end{verbatim}

\begin{Shaded}
\begin{Highlighting}[]
\CommentTok{\# Uloha 3}
\FunctionTok{qexp}\NormalTok{(}\FloatTok{0.05}\NormalTok{, }\AttributeTok{rate =} \DecValTok{1}\SpecialCharTok{/}\DecValTok{2}\NormalTok{, }\AttributeTok{lower.tail =}\NormalTok{ F)}
\end{Highlighting}
\end{Shaded}

\begin{verbatim}
## [1] 5.991465
\end{verbatim}

\begin{Shaded}
\begin{Highlighting}[]
\CommentTok{\# Nakreslite histogram, prelozte hustotu}
\NormalTok{xx }\OtherTok{\textless{}{-}} \FunctionTok{rexp}\NormalTok{(}\DecValTok{500}\NormalTok{, }\AttributeTok{rate =} \DecValTok{1}\SpecialCharTok{/}\DecValTok{2}\NormalTok{) }\CommentTok{\# pre histogram}
\NormalTok{xxx }\OtherTok{\textless{}{-}} \FunctionTok{seq}\NormalTok{(}\DecValTok{0}\NormalTok{, }\DecValTok{5}\NormalTok{, }\FloatTok{0.01}\NormalTok{) }\CommentTok{\# pre kreslenie hustoty}
\FunctionTok{hist}\NormalTok{(xx, }\AttributeTok{freq =}\NormalTok{ F) }\CommentTok{\# freq F na zmenu mierky na pravdepodobnostnu}
\FunctionTok{lines}\NormalTok{(xxx, }\FunctionTok{dexp}\NormalTok{(xxx, }\AttributeTok{rate =} \DecValTok{1}\SpecialCharTok{/}\DecValTok{2}\NormalTok{), }\AttributeTok{col =} \StringTok{\textquotesingle{}purple\textquotesingle{}}\NormalTok{)}
\end{Highlighting}
\end{Shaded}

\includegraphics{cv2_files/figure-latex/unnamed-chunk-1-6.pdf}

\end{document}
